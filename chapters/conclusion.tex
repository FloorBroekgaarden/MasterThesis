\chapter{Conclusion}
\label{ch:conclusion}
%
%Using power colours, black hole and neutron star \acp{LMXB} have been shown to follow a similar power spectral evolution, providing fresh evidence for a common variability origin from an accretion disk. While significant variability in power spectra can occur over short time scales, power colour tracks show the broad band evolution is relatable and model-independent. Diverging power colour paths additionally allow for the rapid classification of neutron star and black hole \acp{LMXB} in the hard states. Further evidence for common behaviour can be found in spectral states, with a link being found between neutron star power colours and the various canonical atoll states, similar to the analogous behaviour discovered for black hole spectral states \citep{heil2015power}. This allows for a comparison between systems with a similar geometry. \\
%
%The effect of a number of other \ac{LMXB} properties on timing properties were also tested. Speculative signs of an inclination dependence of neutron star hardness are found, however they run counter to expectations and require more research. A clear division could be made in the \ac{PCC}~diagram between the population of atoll and Z sources, with the latter remaining firmly in the soft states. Cyg- and Sco-like Z sources showed a tentative difference in power colour spread, however this could be due to observational biases. The effects of pulsations on power colours were also investigated. While bursters showed no particular effects on power colours, a tentative relation between the spin frequency of pulsars and the associated power colour tracks was found. This suggests that strong magnetic fields could affect the timing properties of neutron star \acp{LMXB}. \\
%
%The research conducted in this project could benefit from further research into the effect of broader power colour frequency bands, and into the necessity of various extraction settings, to ensure optimal use of data. While beyond the scope of this project, research into the relationship between power colours and \QPOs would be fascinating, potentially allowing the evolution of \QPOs to be linked to the broader spectral evolution. A wide range of other parameters from mass to the presence of bursts could also be provide new insights into the similarities between black hole and neutron star \acp{LMXB} and effects due to the difference in compact object.